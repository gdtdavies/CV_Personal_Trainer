% This section should present the findings of your work, and discuss them in the context of your original aims & objectives.  For software-oriented projects, how well does it meet your original requirements? Provide data where possible, e.g., results of user testing, performance measures, etc. For research-oriented projects, you should present the data in an appropriate format (tables, charts, visualisations) and provide a critical discussion around the results that provides insight and direction for future work.

%This chapter should present the results of your work, be they in the form of a software product, experimental research findings, or both. You should also link these results to your original objectives.

% goal ~2.5k words
% actual 1057 words

\section{Performance Evaluation}
    As there is no ground truth to compare the detections to, the evaluation of the detections will have to be qualitative rather than quantitative. That being said, the detections seemed to struggle when the lighting was too unfavorable such as when in a very sunny environment, and they also struggled to detect the user for the first time when their entire body wasn't visible, once a detection was made the users where able to get closer to the camera hiding parts of their bodies out of frame. The detections function both with the user facing the camera and with the user looking perpendicularly to the camera's line of sight. The accuracy of the angle calculation does seem to be lower when the user is facing the camera directly as the angle calculations do not currently use the z-axis (depth).\\
    When all the relevant joints were in frame, the repetition counting was very accurate when testing, but when one or more of the joints went out of frame, a large number of false positives were counted. This can be solved by instructing the user to step further away from the camera or change the angle of the camera so that all the joints are in frame.\\
    The performance of the database management system is sufficiently capable for the current scale and scope of the program, this will need to be revisited if the program reaches a much larger userbase.\\
    The performances of the GUI and user experience are described in section \ref{sec:feedback}
    
\section{Comparison with Existing Solutions}
    While some virtual coaching solutions already exist, such as the Apple Watch and Samsung Galaxy Watch, it would be interesting to compare existing virtual coaching against computer vision-based systems. Both of the wearables do great jobs at health and fitness tracking, from heart rate monitoring and ECG, through monitoring blood oxygen levels to sleep tracking. They offer real-time feedback during workouts, automatic workout detection, integration with third-party fitness apps, making them quite versatile tools for various types of exercises. These watches, however, are limited as they are unable to detect the exact position and pose of the user, making them less efficient at detecting poor form. They provide no visual feedback at all during the exercise; only auditory and haptic feedback. Moreover, they can only automatically detect exercises that involve arm movement. This is quite opposite to computer vision-based systems, which work best for movement analysis and real-time feedback on the form and technique of exercises. Such information cannot be provided with regular wrist-based wearables. While smartwatches may be portable and convenient, computer vision-based systems can be much more accurate and richer in the movement analysis, hence highly effective for detailed coaching and corrections in form. The primary feature that the watches are capable of that visual methods can't is heart rate monitoring of vital signs including the measuring of blood oxygen, body temperature, and sleeping patterns. In theory the wearables are well placed to judge arm movements, but due to the fact that they are only worn one a single wrist, the user is constrained to bilateral exercises or to move the wearable to the other wrist to perform unilateral sets. The watch being positioned on one side of the user also allows the user to have poor form on the side that is not tracked. One advantage that wearables have over computer vision based methods is that it is not prone to poor camera placement and angles which would limit the effectiveness of the detections and quality of the recommendations.
\section{User Feedback and Usability Testing \label{sec:feedback}}
    %get Mum, Dad, and Connie to test  
    The login and register are reported as easy to use and simple to understand, they are familiar because they are similar to the login/register commonly found on websites. The history page received very positive feedback, with the user stating that it provides a quick summary that will be good for future workouts to know at what weight to start at, avoiding having to do sets at weights that are potentially too low to be effective or so high that they risk injury. They also stated that they appreciated the two list format. There was also very positive comments on the instant visual feedback provided by the camera and the visible skeleton of the body parts in motion during the exercise.\\
    The main complaints came from the individual workout pages. There was confusion and annoyance with the rest timer, as the desired rest time would have to be re-entered after every set. The ease of understanding of the GUI was lacking, with no clear distinction between active time and rest time. The users expressed a desire for clearer instructions from the chatbox as at the time the only instruction was to set the rest timer if it had not already been done. The legibility at a distance of the text both on the camera side and the controls side was said to be poor, in particular  the repetition counter at the top right corner.\\
    Many recommendations were suggested and have been taken note of for improvement, such as moving the repetition counter to be on top of the camera feed and in larger font, adding more descriptive instructions to the chatbox, and having a "start set" button to make the application easier to understand.
\section{Limitations of the Project}
    Due to the very short time-frame in which this project needed to be planned, designed, programmed, and written up, the scope of this project was very restricted. As mentioned in previous sections of this report, there are many possible improvements that I would have liked to be able to implement but was not able due to lack of time. This includes having functionality for more than the two arm exercises that were implemented, having a more functional chatbox that would give feedback to the users specifically about their form, and not having the time to train, nor to obtain user feedback from more users. I also lacked the ethical approval for data collection for a deep learning model for repetition counting and form recommendations, which is something that would have greatly improved this project. As I am a single student developer and not a team of professional software engineers, this project was never going to result in a completed program with all the functionality required for public use, this is why this project acts more as a proof of concept and a foundation on which more research can based on. 