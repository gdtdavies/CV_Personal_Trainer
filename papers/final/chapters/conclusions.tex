% This is another relatively short chapter that is an opportunity to reflect on the project as a whole. Discuss the limitations and successes of the project, highlighting opportunities for future work. 

%This chapter should summarise the work you have undertaken by revisiting your project aims and objectives and what you have achieved to meet them. You should include a summarised response to any additional research questions you addressed. It is also important to discuss the limitations of the work and implications for future research in the space based on what you have done.

% goal ~200 words
% actual 363 words

\section{Summary of Project}
    This Project sought to create a computer-vision based training coach for computerised physical training, by creating a computer program capable of making recommendations on physical exercises and form to its users based on camera input where the users' body and movements are detectable.
    Implementing this project involved the creation of an application that is capable of the aforementioned tasks and embedding it into a graphical user interface that is easy to navigate and provides the necessary information to the user. A simple backend was attached allowing for a login system with workout history and mood tracking.
    
\section{Contributions to the Field}
    This project contributed to the field of both computer science and more specifically human pose estimation but also to the fields of sports psychology and physical training. By creating this program, many research avenues have been facilitated and opened up in terms of studying the effect of exercise on mood and mental health and on how computer vision, HPE, and robotics in general can help in this sector.   
    
\section{Practical Implications}
    This project acts as a baseline from which further research can be make. Allowing for experts and researchers in the field of sports and physical health to utilise computer vision technologies to perform studies on not only the impact of exercise on mood and mental health but also many more questions. This program, if improved and iterated upon, could have profound positive implications on public health in terms of getting the general population to exercise more and improve our understanding of certain workouts and how they are performed optimally.
    
\section{Recommendations for Future Work}
    There is still much more that can be added to this program. More types of exercises, better recommendation system, using deep learning methods to train a recommendation model, and much more in terms of the application. Another recommendation for future work would be to port the program as a smartphone application, this will obviously require work on server side implementations and allowing models to either be much smaller while keeping their accuracy or be able to run on cloud computing.
\section{Final Thoughts}
    Overall, even though I would have liked to have been able to implement more functionality into my application, I believe that this have laid a good foundation for future research.