% This is samplepaper.tex, a sample chapter demonstrating the
% LLNCS macro package for Springer Computer Science proceedings;
% Version 2.20 of 2017/10/04
%
\documentclass[runningheads]{llncs}
%
\usepackage{graphicx}
% Used for displaying a sample figure. If possible, figure files should
% be included in EPS format.
\usepackage[english]{babel}
\usepackage[utf8]{inputenc}
\usepackage{amsmath}
\usepackage{csquotes}% Recommended

\usepackage[style=authoryear-ibid,backend=biber]{biblatex}

\addbibresource{ref.bib}% Syntax for version >= 1.2
%
% If you use the hyperref package, please uncomment the following line
% to display URLs in blue roman font according to Springer's eBook style:
% \renewcommand\UrlFont{\color{blue}\rmfamily}

\begin{document}
%
\title{A computer-vision based training coach for computerized physical training}
%
%\titlerunning{Abbreviated paper title}
% If the paper title is too long for the running head, you can set
% an abbreviated paper title here
%
\author{George Davies\orcidID{0009-0008-4132-5676}}
%
\authorrunning{G. Davies}
% First names are abbreviated in the running head.
% If there are more than two authors, 'et al.' is used.
%
\institute{University of Lincoln, UK 
\email{27421138@students.lincoln.ac.uk}}
%
\maketitle              % typeset the header of the contribution
%

% The project proposal should be aligned with your programme of study and be produced following discussions with your supervisor around the suitability and scope of the research you will undertake. The work you will do for the proposal will help you frame the independent research you will do for the module's duration. As with all assessments for the module, you will be expected to undertake research yourself and make informed decisions independently that support your project’s research aims and objectives. The recommended structure for the proposal document can be found under the “Additional Information” section of this briefing document. Your submission should conform to one of the suppled templates and be submitted as a PDF as per school policy. The templates can be downloaded from Blackboard under Assessments>Assessment Documents>Assessment Item 1.

% Ethical approval (where applicable)
% If your project is likely to involve human participants in a low-risk capacity (e.g., a user study) then you must complete an ethics application via LEAS. Your LEAS application must be submitted to the LEAS service by the same deadline as the proposal. You should discuss this with your supervisor before completing your ethics application, your supervisor will provide support for this. Consideration of ethical issues must also be evidenced in the ‘Ethical Considerations’ section of the project proposal.

\section{Introduction}
    In many stories of science-fiction, there are robots that are intelligent, borderline human in the way they talk, the way they walk, and the way they interact with their environments. These kinds of robots seem to be creatures that won't exist any time soon, and they probably won't, but with every advancement in the field of robotics and autonomous systems, we step closer to that reality. One key sense that helps a robot communicate and understand its environment is sight, and the area of robotics that strives towards the goal of giving machines the sense of sight is computer vision. 

    \subsection{Area of Research}
        Human Pose Estimation (HPE) is an area of research within computer vision that aims to teach robots how to make sense of the human form and the motions it is capable of performing. It involves the identification and classification of the joints in the human body, capturing a set of coordinates for each joint, known as a key point, that can describe the pose of a person. HPE has a wide set of uses in many fields: In games, with motion capture technologies reliant on HPE, it allows developers to code program more realistic and fluid character movements. In healthcare, healthcare providers can monitor a patient's movements and detect any abnormalities. Augmented reality, allows the user to interact with the digital content in more natural and intuitive ways with gestures. And finally, the use-case that is the primary focus of this project, is sports training. HPE can be used to analyse a user's performance, identify areas for improvement, and develop personalised training programs based on the physical level of the user. For example, HPE could be used to analyse a runner's form, e.g. How straight is their back? What part of the foot they are landing on? Are they leaning more to one side?..., and providing feedback on how to improve their technique. HPE can be used to collect data about any exercises where the movement of the body is vital to its effectiveness.
    
    \subsection{Relevance to the Degree Programme}
        I am enrolled in the MSc programme Robotics and Autonomous Systems at the Univerity of Lincoln through the AgriFoRwArdS CDT.\@ Throughout the first two semesters, I studied the principles of robotics, artificial intelligence, machine learning, and computer vision. Principles from all of these subjects are applied within the area of research on human pose estimation. As to my affiliation with the AgriFoRwArdS CDT, their focus is on the production and use of AI, ML, and CV applications to help the farming and agricultural industries, as Lincolnshire is an agricultural region of the UK.\@ Human pose estimation has previously been used in agritech applications \parencite{app12168160}, with its ability to facilitate human-robot interactions in the field for fruit picking, robotic carts will follow the worker through the field to hold the produce and take it away once full. HPE allows these robots to understand gesture commands the worker may give it, and gives the robot an understanding of humans that allows it to find and follow them without driving into them.
        


    \subsection{Background of the Topic}
        When computer vision gained popularity in the late 1960s and early 1970s, HPE research had its start. Scientists first focused on basic problems like as shape analysis, object recognition, and visual understanding. As computer vision developed, HPE became a stand-alone area of study \parencite{Roboflow}. Historically, HPE was frequently described probabilistically to account for likely inference ambiguities. Since deep learning has been more widely used, the focus has switched to end-to-end trainable models because of their ability to extract intricate patterns and postures from data. Traditionally, computer vision systems have assessed an object's or person's posture by geometric calculations and feature-based techniques. But, the biggest developments in HPE came with the advent of deep neural networks, convolutional neural networks, and computer vision. The field has advanced considerably in spite of these challenges, and more recent techniques that make use of properly designed neural networks may provide amazing results in challenging scenarios involving a large number of, perhaps veiled, interacting individuals \parencite{liu2018recognizing}. Now that these detections have the necessary technology and are sufficiently precise, they may be employed for commercial purposes. It also offers a wealth of new application potential and signifies a major change in HPE's overall direction.



\section{Aims and Objectives}
%You should formulate these clearly, explaining what problems are to be explored and why they are worth exploring. You should have a single overarching aim that comprises your overall research question.
    \subsection{Issues to Explore}

    \subsection{Motivation}

    \subsection{End Goal}


\section{Literature Survey}
%Provide a short literature survey based on key pieces of work. For example, 3 or 4 relevant papers on the topic of interest will suffice.

    \subsection{Paper 1}

    \subsection{Paper 2}

    \subsection{Paper 3}

    \subsection{Paper 4}


\section{Research Methods}
% Briefly discuss the research methods that may be appropriate for the proposed research. For example, are they quantitative or qualitative and how would you deploy them? Will you use any publicly available data?


\section{Ethical Considerations}
% In consultation with your supervisor, you must evidence any ethical issues in this section and clearly identify if you have submitted an ethical application via LEAS.


\section{Project Plan and Risk Analysis}
% A documented project plan spanning the full timeframe of the project. Timescales and milestones/deliverables should be provided for each of the project objectives. This may take the form of a Gantt chart, with granularity of no more than one week. The risk analysis should identify and explain specific risks, their likelihood, and assessed impact, with a mitigation strategy for each. Generic risks (e.g., illness, loss of data, IT problems etc.) are common to all projects and should NOT be included here.
    \subsection{Project Plan}

    \subsection{Risk Analysis}

\newpage
\printbibliography
\centering{\large{\textbf{Word Count: 765}}}
\end{document}
