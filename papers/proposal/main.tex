% This is samplepaper.tex, a sample chapter demonstrating the
% LLNCS macro package for Springer Computer Science proceedings;
% Version 2.20 of 2017/10/04
%
\documentclass[runningheads]{llncs}
%
\usepackage{graphicx}
% Used for displaying a sample figure. If possible, figure files should
% be included in EPS format.
\usepackage[english]{babel}
\usepackage[utf8]{inputenc}
\usepackage{amsmath}
\usepackage{csquotes}% Recommended

\usepackage[style=authoryear-ibid,backend=biber]{biblatex}

\addbibresource{ref.bib}% Syntax for version >= 1.2
%
% If you use the hyperref package, please uncomment the following line
% to display URLs in blue roman font according to Springer's eBook style:
% \renewcommand\UrlFont{\color{blue}\rmfamily}

\begin{document}
%
\title{A computer-vision based training coach for computerized physical training}
%
%\titlerunning{Abbreviated paper title}
% If the paper title is too long for the running head, you can set
% an abbreviated paper title here
%
\author{George Davies\orcidID{0009-0008-4132-5676}}
%
\authorrunning{G. Davies}
% First names are abbreviated in the running head.
% If there are more than two authors, 'et al.' is used.
%
\institute{University of Lincoln, UK 
\email{27421138@students.lincoln.ac.uk}}
%
\maketitle              % typeset the header of the contribution
%

% The project proposal should be aligned with your programme of study and be produced following discussions with your supervisor around the suitability and scope of the research you will undertake. The work you will do for the proposal will help you frame the independent research you will do for the module's duration. As with all assessments for the module, you will be expected to undertake research yourself and make informed decisions independently that support your project’s research aims and objectives. The recommended structure for the proposal document can be found under the “Additional Information” section of this briefing document. Your submission should conform to one of the suppled templates and be submitted as a PDF as per school policy. The templates can be downloaded from Blackboard under Assessments>Assessment Documents>Assessment Item 1.

% Ethical approval (where applicable)
% If your project is likely to involve human participants in a low-risk capacity (e.g., a user study) then you must complete an ethics application via LEAS. Your LEAS application must be submitted to the LEAS service by the same deadline as the proposal. You should discuss this with your supervisor before completing your ethics application, your supervisor will provide support for this. Consideration of ethical issues must also be evidenced in the ‘Ethical Considerations’ section of the project proposal.

\section{Introduction}
%You should indicate the area/discipline where the research sits and how it aligns with the degree programme you are enrolled on. You should also provide a brief discussion on the background and issues of the research you are interested in undertaking.

% Absolutely, here's a more detailed breakdown of each section:

% **1. Area of Research**
%    - Begin by introducing the field of Human Pose Estimation (HPE). You could say something like, "Human Pose Estimation is a significant area of research within the field of computer vision. It involves the identification and classification of the joints in the human body, capturing a set of coordinates for each joint, known as a key point, that can describe a pose of a person."
%    - Discuss the importance of HPE in computer vision, "The ability to accurately estimate human poses has wide-ranging implications for various applications, including gaming, healthcare, augmented reality, and sports training."
%    - Highlight the current challenges in this field, "Despite the advancements in HPE, there are still challenges to overcome, such as handling occlusions, estimating poses in complex environments, and improving real-time performance."

% **2. Relevance to Degree (Robotics and Autonomous Systems)**
%    - Discuss the relevance of HPE to your degree, "In the context of Robotics and Autonomous Systems, HPE plays a crucial role. It enables robots to understand and interact with the human world, which is essential for tasks such as human-robot collaboration, assistive technology, and autonomous navigation."
%    - Talk about how your project could contribute to advancements in your field, "By developing a computer-vision based training coach using HPE, we can push the boundaries of what's possible in Robotics and Autonomous Systems. This project could lead to more interactive and responsive robots that can adapt to human movements and provide real-time feedback."

% **3. Background of the Topic**
%    - Provide a brief history of HPE, "The field of HPE has evolved significantly over the years. Early methods relied on hand-crafted features and shallow learning models. However, with the advent of deep learning, the focus has shifted towards end-to-end trainable models that can learn to recognize complex patterns and poses from data."
%    - Discuss the different techniques used in HPE, "There are two main approaches to HPE: classical methods and deep learning-based methods. Classical methods typically involve steps like detecting body parts, representing them as graphical models, and then inferring the pose. On the other hand, deep learning-based methods leverage convolutional neural networks to directly learn the mapping from input images to poses."
%    - Mention some key research papers or breakthroughs in this field, "There have been several key breakthroughs in HPE. For instance, the introduction of large-scale datasets like MPII Human Pose and COCO, and the development of models like OpenPose and DeepPose have significantly advanced the field."

% Imagine a world where technology can understand and interpret human movement as naturally as we do. This is the fascinating world of Human Pose Estimation (HPE), a significant area of research within the field of computer vision. It’s like teaching a computer to see and understand the poetry of human motion. HPE is about capturing the essence of human movement by identifying and classifying the joints in the human body. Each joint is captured as a key point, a set of coordinates that together describe a pose. It’s like creating a digital skeleton that mirrors our movements. The potential applications of HPE are vast and exciting, from gaming and healthcare to augmented reality and sports training. But like any frontier, it comes with its own set of challenges. Handling occlusions, estimating poses in complex environments, and improving real-time performance are just a few of the hurdles researchers are striving to overcome.

% Now, let’s bring robots into the picture. In the realm of Robotics and Autonomous Systems, HPE is more than just a tool; it’s a bridge that connects robots to the human world. It enables robots to understand and interact with us, paving the way for human-robot collaboration, assistive technology, and autonomous navigation. This project, a computer-vision based training coach using HPE, is not just a research endeavor. It’s an opportunity to push the boundaries of what’s possible in Robotics and Autonomous Systems. It’s about creating robots that are not just machines, but interactive and responsive companions that can adapt to human movements and provide real-time feedback.

% The journey of HPE has been a remarkable one. It started with early methods that relied on hand-crafted features and shallow learning models. But as the field of artificial intelligence evolved, so did HPE. The advent of deep learning marked a paradigm shift, with end-to-end trainable models that could learn to recognize complex patterns and poses from data. Today, we have two main approaches to HPE: classical methods and deep learning-based methods. While classical methods involve a series of steps from detecting body parts to inferring the pose, deep learning-based methods use convolutional neural networks to directly learn the mapping from input images to poses. The field has seen several breakthroughs, from the introduction of large-scale datasets like MPII Human Pose and COCO, to the development of models like OpenPose and DeepPose. Each breakthrough brings us one step closer to making computers understand human movement as naturally as we do.

%Gaming: In the gaming industry, HPE can be used to create more immersive and interactive experiences. For instance, motion capture technology, which relies heavily on HPE, allows game developers to create more realistic and fluid character movements. Additionally, HPE can enable gesture-based controls, allowing players to interact with games using their body movements.

% Healthcare: In healthcare, HPE can be used in a variety of ways. For instance, it can be used for patient monitoring, allowing healthcare providers to track a patient’s movements and detect any abnormalities. It can also be used in physical therapy to monitor a patient’s progress and ensure they are performing exercises correctly. Furthermore, HPE can play a crucial role in developing assistive technologies for individuals with physical disabilities.

% Augmented Reality (AR): AR involves overlaying digital information onto the real world, and HPE can enhance this process by allowing the digital content to interact with the user in a more natural and intuitive way. For instance, AR applications can use HPE to understand the user’s gestures and respond accordingly, creating a more immersive and interactive experience.

% Sports Training: In sports training, HPE can be used to analyze an athlete’s performance, identify areas for improvement, and develop personalized training programs. For instance, a running coach could use HPE to analyze a runner’s form and provide feedback on how to improve their technique. Similarly, a yoga instructor could use HPE to ensure that students are performing poses correctly, providing real-time feedback and adjustments.
    In many stories of science-fiction, there are robots that are intelligent, borderline human in the way they talk, the way they walk, and the way they interact with their environments. These kinds of robots seem to be creatures that won't exist any time soon, and they probably won't, but with every advancement in the field of robotics and autonomous systems, we step closer to that reality. One key sense that helps a robot communicate and understand its environment is sight, and the area of robotics that strives towards the goal of giving machines the sense of sight is computer vision. 

    \subsection{Area of Research}
        Human Pose Estimation (HPE) is an area of research within computer vision that aims to teach robots how to make sense of the human form and the motions it is capable of performing. It involves the identification and classification of the joints in the human body, capturing a set of coordinates for each joint, known as a key point, that can describe the pose of a person. HPE has a wide set of uses in many fields: In games, with motion capture technologies reliant on HPE, it allows developers to code program more realistic and fluid character movements. In healthcare, healthcare providers can monitor a patient's movements and detect any abnormalities. Augmented reality, allows the user to interact with the digital content in more natural and intuitive ways with gestures. And finally, the use-case that is the primary focus of this project, is sports training. HPE can be used to analyse a user's performance, identify areas for improvement, and develop personalised training programs based on the physical level of the user. For example, HPE could be used to analyse a runner's form, e.g. How straight is their back? What part of the foot they are landing on? Are they leaning more to one side?..., and providing feedback on how to improve their technique. HPE can be used to collect data about any exercises where the movement of the body is vital to its effectiveness.
    
    \subsection{Relevance to the Degree Programme}
        I am enrolled in the MSc programme Robotics and Autonomous Systems at the Univerity of Lincoln through the AgriFoRwArdS CDT.\@ Throughout the first two semesters, I studied the principles of robotics, artificial intelligence, machine learning, and computer vision. Principles from all of these subjects are applied within the area of research on human pose estimation. As to my affiliation with the AgriFoRwArdS CDT, their focus is on the production and use of AI, ML, and CV applications to help the farming and agricultural industries, as Lincolnshire is an agricultural region of the UK.\@ Human pose estimation has previously been used in agritech applications \parencite{app12168160}, with its ability to facilitate human-robot interactions in the field for fruit picking, robotic carts will follow the worker through the field to hold the produce and take it away once full. HPE allows these robots to understand gesture commands the worker may give it, and gives the robot an understanding of humans that allows it to find and follow them without driving into them.
        


    \subsection{Background of the Topic}
        When computer vision gained popularity in the late 1960s and early 1970s, HPE research had its start. Scientists first focused on basic problems like as shape analysis, object recognition, and visual understanding. As computer vision developed, HPE became a stand-alone area of study \parencite{Roboflow}. Historically, HPE was frequently described probabilistically to account for likely inference ambiguities. Since deep learning has been more widely used, the focus has switched to end-to-end trainable models because of their ability to extract intricate patterns and postures from data. Traditionally, computer vision systems have assessed an object's or person's posture by geometric calculations and feature-based techniques. But, the biggest developments in HPE came with the advent of deep neural networks, convolutional neural networks, and computer vision. The field has advanced considerably in spite of these challenges, and more recent techniques that make use of properly designed neural networks may provide amazing results in challenging scenarios involving a large number of, perhaps veiled, interacting individuals \parencite{liu2018recognizing}. Now that these detections have the necessary technology and are sufficiently precise, they may be employed for commercial purposes. It also offers a wealth of new application potential and signifies a major change in HPE's overall direction.



\section{Aims and Objectives}
%You should formulate these clearly, explaining what problems are to be explored and why they are worth exploring. You should have a single overarching aim that comprises your overall research question.
    \subsection{Issues to Explore}
        ultrices gravida dictum fusce ut placerat orci nulla pellentesque dignissim enim sit amet venenatis urna cursus eget nunc scelerisque viverra mauris in aliquam sem fringilla ut morbi tincidunt augue interdum velit euismod in pellentesque massa placerat duis ultricies lacus sed turpis tincidunt id aliquet risus feugiat in ante metus dictum at tempor commodo ullamcorper a lacus vestibulum sed arcu non odio euismod lacinia at quis risus sed vulputate odio ut enim blandit volutpat maecenas volutpat blandit aliquam etiam erat velit scelerisque in dictum non consectetur a erat nam at lectus urna duis convallis convallis tellus id interdum velit laoreet id donec ultrices tincidunt arcu non sodales neque sodales

    \subsection{Motivation}
        ut etiam sit amet nisl purus in mollis nunc sed id semper risus in hendrerit gravida rutrum quisque non tellus orci ac auctor augue mauris augue neque gravida in fermentum et sollicitudin ac orci phasellus egestas tellus rutrum tellus pellentesque eu tincidunt tortor aliquam nulla facilisi cras fermentum odio eu feugiat pretium nibh ipsum consequat nisl vel pretium lectus quam id leo in vitae turpis massa sed elementum tempus egestas sed sed risus pretium quam vulputate dignissim suspendisse in est ante in nibh mauris cursus mattis molestie a iaculis at erat pellentesque adipiscing commodo elit at imperdiet dui accumsan sit amet nulla facilisi morbi tempus iaculis urna id volutpat lacus laoreet non

    \subsection{End Goal}
        curabitur gravida arcu ac tortor dignissim convallis aenean et tortor at risus viverra adipiscing at in tellus integer feugiat scelerisque varius morbi enim nunc faucibus a pellentesque sit amet porttitor eget dolor morbi non arcu risus quis varius quam quisque id diam vel quam elementum pulvinar etiam non quam lacus suspendisse faucibus interdum posuere lorem ipsum dolor sit amet consectetur adipiscing elit duis tristique sollicitudin nibh sit amet commodo nulla facilisi nullam vehicula ipsum a arcu cursus vitae congue mauris


\section{Literature Survey}
%Provide a short literature survey based on key pieces of work. For example, 3 or 4 relevant papers on the topic of interest will suffice.

    \subsection{Paper 1}
        egestas sed tempus urna et pharetra pharetra massa massa ultricies mi quis hendrerit dolor magna eget est lorem ipsum dolor sit amet consectetur adipiscing elit pellentesque habitant morbi tristique senectus et netus et malesuada fames ac turpis egestas integer eget aliquet nibh praesent tristique magna sit amet purus gravida quis blandit turpis cursus in hac habitasse platea dictumst quisque sagittis purus sit amet volutpat consequat mauris nunc congue nisi vitae suscipit tellus mauris a diam maecenas sed enim ut sem viverra aliquet eget sit amet tellus cras adipiscing enim eu turpis egestas pretium aenean pharetra magna ac placerat vestibulum lectus mauris ultrices eros in cursus turpis massa tincidunt dui ut ornare lectus sit amet est placerat in egestas erat imperdiet sed euismod nisi porta lorem mollis aliquam ut porttitor leo a diam sollicitudin tempor id eu nisl nunc mi ipsum faucibus vitae aliquet nec ullamcorper sit amet risus nullam eget felis eget nunc lobortis mattis aliquam faucibus purus in massa tempor nec feugiat nisl pretium fusce id velit ut tortor pretium viverra suspendisse potenti nullam ac tortor vitae purus faucibus ornare suspendisse sed nisi lacus sed viverra tellus in hac habitasse platea dictumst vestibulum rhoncus est

    \subsection{Paper 2}
        pellentesque elit ullamcorper dignissim cras tincidunt lobortis feugiat vivamus at augue eget arcu dictum varius duis at consectetur lorem donec massa sapien faucibus et molestie ac feugiat sed lectus vestibulum mattis ullamcorper velit sed ullamcorper morbi tincidunt ornare massa eget egestas purus viverra accumsan in nisl nisi scelerisque eu ultrices vitae auctor eu augue ut lectus arcu bibendum at varius vel pharetra vel turpis nunc eget lorem dolor sed viverra ipsum nunc aliquet bibendum enim facilisis gravida neque convallis a cras semper auctor neque vitae tempus quam pellentesque nec nam aliquam sem et tortor consequat id porta nibh venenatis cras sed felis eget velit aliquet sagittis id consectetur purus ut faucibus pulvinar elementum integer enim neque volutpat ac tincidunt vitae semper quis lectus nulla at volutpat diam ut venenatis tellus in metus vulputate eu scelerisque felis imperdiet proin fermentum leo vel orci porta non pulvinar neque laoreet suspendisse interdum consectetur libero id faucibus nisl tincidunt eget nullam non nisi est sit amet facilisis magna etiam tempor orci eu lobortis

    \subsection{Paper 3}
        elementum nibh tellus molestie nunc non blandit massa enim nec dui nunc mattis enim ut tellus elementum sagittis vitae et leo duis ut diam quam nulla porttitor massa id neque aliquam vestibulum morbi blandit cursus risus at ultrices mi tempus imperdiet nulla malesuada pellentesque elit eget gravida cum sociis natoque penatibus et magnis dis parturient montes nascetur ridiculus mus mauris vitae ultricies leo integer malesuada nunc vel risus commodo viverra maecenas accumsan lacus vel facilisis volutpat est velit egestas dui id ornare arcu odio ut sem nulla pharetra diam sit amet nisl suscipit adipiscing bibendum est ultricies integer quis auctor elit sed vulputate mi sit amet mauris commodo quis imperdiet massa tincidunt nunc pulvinar sapien et ligula ullamcorper malesuada proin libero nunc consequat interdum varius sit amet mattis vulputate enim nulla aliquet porttitor lacus luctus accumsan tortor posuere ac ut consequat semper viverra nam libero justo laoreet sit amet cursus sit amet dictum sit amet justo donec enim diam vulputate ut pharetra sit amet aliquam id diam maecenas ultricies mi eget mauris

    \subsection{Paper 4}
       pharetra et ultrices neque ornare aenean euismod elementum nisi quis eleifend quam adipiscing vitae proin sagittis nisl rhoncus mattis rhoncus urna neque viverra justo nec ultrices dui sapien eget mi proin sed libero enim sed faucibus turpis in eu mi bibendum neque egestas congue quisque egestas diam in arcu cursus euismod quis viverra nibh cras pulvinar mattis nunc sed blandit libero volutpat sed cras ornare arcu dui vivamus arcu felis bibendum ut tristique et egestas quis ipsum suspendisse ultrices gravida dictum fusce ut placerat orci nulla pellentesque dignissim enim sit amet venenatis urna cursus eget nunc scelerisque viverra mauris in aliquam sem fringilla ut morbi tincidunt augue interdum velit euismod in pellentesque massa placerat duis ultricies lacus sed turpis tincidunt id aliquet risus feugiat in ante metus dictum at tempor commodo ullamcorper a lacus vestibulum sed arcu non odio euismod lacinia at quis risus sed vulputate odio ut enim blandit volutpat maecenas volutpat blandit aliquam etiam erat velit scelerisque in dictum non consectetur


\section{Research Methods}
% Briefly discuss the research methods that may be appropriate for the proposed research. For example, are they quantitative or qualitative and how would you deploy them? Will you use any publicly available data?
    semper quis lectus nulla at volutpat diam ut venenatis tellus in metus vulputate eu scelerisque felis imperdiet proin fermentum leo vel orci porta non pulvinar neque laoreet suspendisse interdum consectetur libero id faucibus nisl tincidunt eget nullam non nisi est sit amet facilisis magna etiam tempor orci eu lobortis elementum nibh tellus molestie nunc non blandit massa enim nec dui nunc mattis enim ut tellus elementum sagittis vitae et leo duis ut diam quam nulla porttitor massa id neque aliquam vestibulum morbi blandit cursus risus at ultrices mi tempus imperdiet nulla malesuada pellentesque elit eget gravida cum sociis natoque penatibus et magnis dis parturient montes nascetur ridiculus mus mauris vitae ultricies leo integer malesuada nunc vel risus commodo viverra maecenas accumsan lacus vel facilisis volutpat est velit egestas dui id ornare arcu odio ut sem nulla pharetra diam sit amet nisl suscipit adipiscing bibendum est ultricies integer quis auctor elit sed vulputate mi sit amet mauris commodo quis imperdiet massa tincidunt nunc pulvinar sapien et ligula ullamcorper malesuada proin libero nunc consequat interdum varius sit amet mattis vulputate enim nulla aliquet porttitor lacus luctus accumsan tortor posuere ac ut consequat semper viverra nam libero justo laoreet sit amet cursus sit amet dictum sit amet justo donec enim diam vulputate ut pharetra sit amet aliquam id diam maecenas ultricies mi eget mauris pharetra et ultrices neque ornare aenean euismod elementum nisi quis eleifend quam adipiscing vitae proin sagittis nisl rhoncus mattis rhoncus urna neque viverra justo nec ultrices dui sapien eget


\section{Ethical Considerations}
% In consultation with your supervisor, you must evidence any ethical issues in this section and clearly identify if you have submitted an ethical application via LEAS.
    semper quis lectus nulla at volutpat diam ut venenatis tellus in metus vulputate eu scelerisque felis imperdiet proin fermentum leo vel orci porta non pulvinar neque laoreet suspendisse interdum consectetur libero id faucibus nisl tincidunt eget nullam non nisi est sit amet facilisis magna etiam tempor orci eu lobortis elementum nibh tellus molestie nunc non blandit massa enim nec dui nunc mattis enim ut tellus elementum sagittis vitae et leo duis ut diam quam nulla porttitor massa id neque aliquam vestibulum morbi blandit cursus risus at ultrices mi tempus imperdiet nulla malesuada pellentesque elit eget gravida cum sociis natoque penatibus et magnis dis parturient montes nascetur ridiculus mus mauris vitae ultricies leo integer malesuada nunc vel risus commodo viverra maecenas accumsan lacus vel facilisis volutpat est velit egestas dui id ornare arcu odio ut sem nulla pharetra diam sit amet nisl suscipit adipiscing bibendum est ultricies integer quis auctor elit sed vulputate mi sit amet mauris commodo quis imperdiet massa tincidunt nunc pulvinar sapien et ligula ullamcorper malesuada proin libero nunc consequat interdum varius sit amet mattis vulputate enim nulla aliquet porttitor lacus luctus accumsan tortor posuere ac ut consequat semper viverra nam libero justo laoreet sit amet cursus sit amet dictum sit amet justo donec enim diam vulputate ut pharetra sit amet aliquam id diam maecenas ultricies mi eget mauris pharetra et ultrices neque ornare aenean euismod elementum nisi quis eleifend quam adipiscing vitae proin sagittis nisl rhoncus mattis rhoncus urna neque viverra justo nec ultrices dui sapien eget


\section{Project Plan and Risk Analysis}
% A documented project plan spanning the full timeframe of the project. Timescales and milestones/deliverables should be provided for each of the project objectives. This may take the form of a Gantt chart, with granularity of no more than one week. The risk analysis should identify and explain specific risks, their likelihood, and assessed impact, with a mitigation strategy for each. Generic risks (e.g., illness, loss of data, IT problems etc.) are common to all projects and should NOT be included here.
    \subsection{Project Plan}
        scelerisque eu ultrices vitae auctor eu augue ut lectus arcu bibendum at varius vel pharetra vel turpis nunc eget lorem dolor sed viverra ipsum nunc aliquet bibendum enim facilisis gravida neque convallis a cras semper auctor neque vitae tempus quam pellentesque nec nam aliquam sem et tortor consequat id porta nibh venenatis cras sed felis eget velit aliquet sagittis id consectetur purus ut faucibus pulvinar elementum integer enim neque volutpat ac tincidunt vitae semper quis lectus nulla at volutpat diam ut venenatis tellus in metus vulputate eu scelerisque felis imperdiet proin fermentum leo vel orci porta non pulvinar neque laoreet suspendisse interdum consectetur libero id faucibus nisl tincidunt eget nullam non nisi est sit amet facilisis magna etiam tempor orci eu lobortis elementum nibh tellus molestie nunc non blandit massa enim nec dui nunc mattis enim ut tellus elementum sagittis vitae et leo duis ut diam quam nulla porttitor massa id neque aliquam vestibulum morbi blandit 

    \subsection{Risk Analysis}
        cursus risus at ultrices mi tempus imperdiet nulla malesuada pellentesque elit eget gravida cum sociis natoque penatibus et magnis dis parturient montes nascetur ridiculus mus mauris vitae ultricies leo integer malesuada nunc vel risus commodo viverra maecenas accumsan lacus vel facilisis volutpat est velit egestas dui id ornare arcu odio ut sem nulla pharetra diam sit amet nisl suscipit adipiscing bibendum est ultricies integer quis auctor elit sed vulputate mi sit amet mauris commodo quis imperdiet massa tincidunt nunc pulvinar sapien et ligula ullamcorper malesuada proin libero nunc consequat interdum varius sit amet mattis vulputate enim nulla aliquet porttitor lacus luctus accumsan tortor posuere ac ut consequat semper viverra nam libero justo laoreet sit amet cursus sit amet dictum sit amet justo donec enim diam vulputate ut pharetra sit amet aliquam id diam maecenas ultricies mi eget mauris pharetra et ultrices neque ornare aenean

\printbibliography

% pdftotext main.pdf - | wc -w
\textbf{Word Count: TODO -- put word count here}
\end{document}
